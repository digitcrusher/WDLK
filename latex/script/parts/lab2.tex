
\documentclass{article}

% Pakiety umożliwiające dodawanie grafik, formuł matematycznych i polskich znaków
\usepackage{graphicx} % dodawanie obrazków do dokumentu
\usepackage{xcolor}
\usepackage{amsmath, amssymb, amsfonts, amsthm} % tworzenie wyrażeń matematycznych i symboli
\usepackage[T1]{fontenc} % obsługa polskich znaków diakrytycznych
%\usepackage[polish]{babel} % alternatywne ustawienie języka dokumentu na polski

% Definicje środowisk do twierdzeń i dowodów
\newtheorem{thm}{Twierdzenie}
\newtheorem{lem}{Lemat}
\newenvironment{dwd}{\paragraph{Dowód:}}{\hfill$\square$}

% Nagłówek dokumentu z tytułem, autorem i datą
\title{Laboratorium 3}
\author{Maciek Tadej \thanks{Instytut Matematyczny UWr}}
\date{\today}

\begin{document}

% Tworzenie strony tytułowej
\maketitle

%%%%%%%%%%%%%%%%%%%%%%%%%%%%%%%%%%%%%%%%%%%%%%%%%%%%%%%%%%%%%%%%%%%%%%%%%%%%%%%%%%%%%%%%%%%%%%%%%

% Sekcja wstępna z przykładami formatowania tekstu
\section{Formatowanie tekstu}

W tej sekcji pokażemy różne sposoby formatowania tekstu w \LaTeX{}.

\subsection{Styl i wyróżnienia tekstu}
Możemy stosować różne style tekstu:
\begin{itemize}
    \item \textbf{Pogrubienie} za pomocą komendy \texttt{\textbackslash textbf\{...\}}.
    \item \textit{Kursywa} za pomocą komendy \texttt{\textbackslash textit\{...\}}.
    \item \underline{Podkreślenie} za pomocą komendy \texttt{\textbackslash underline\{...\}}.
\end{itemize}

\subsection{Kolorowanie tekstu}
Kolorowanie tekstu wymaga użycia pakietu \texttt{color} lub \texttt{xcolor}. Możemy kolorować tekst na różne kolory, np.:
\begin{itemize}
    \item \textcolor{blue}{Niebieski} za pomocą komendy \texttt{\textbackslash textcolor\{blue\}\{...\}}.
    \item \textcolor{red}{Czerwony} za pomocą komendy \texttt{\textbackslash textcolor\{red\}\{...\}}.
    \item \textcolor{green}{Zielony} za pomocą komendy \texttt{\textbackslash textcolor\{green\}\{...\}}.
\end{itemize}

Dostępne są także bardziej złożone kolory:
\textcolor[rgb]{0.5,0,0.5}{Fioletowy (RGB)}, \textcolor[RGB]{255,128,0}{Pomarańczowy (RGB)}.

\subsection{Zmiana wielkości czcionki}
W \LaTeX{} mamy możliwość zmiany wielkości tekstu:
\begin{itemize}
    \item {\tiny Bardzo mały tekst} za pomocą \texttt{\textbackslash tiny}.
    \item {\scriptsize Mały tekst} za pomocą \texttt{\textbackslash scriptsize}.
    \item {\footnotesize Tekst do przypisów} za pomocą \texttt{\textbackslash footnotesize}.
    \item {\small Mała czcionka} za pomocą \texttt{\textbackslash small}.
    \item {\normalsize Normalna wielkość czcionki} za pomocą \texttt{\textbackslash normalsize}.
    \item {\large Duży tekst} za pomocą \texttt{\textbackslash large}.
    \item {\Large Większy tekst} za pomocą \texttt{\textbackslash Large}.
    \item {\LARGE Jeszcze większy tekst} za pomocą \texttt{\textbackslash LARGE}.
    \item {\huge Bardzo duży tekst} za pomocą \texttt{\textbackslash huge}.
    \item {\Huge Największy tekst} za pomocą \texttt{\textbackslash Huge}.
\end{itemize}

\subsection{Wyrównywanie tekstu}
Aby wyrównać tekst, możemy używać różnych środowisk:
\begin{itemize}
    \item \texttt{center} do centrowania tekstu:
    \begin{center}
        Ten tekst jest wyśrodkowany.
    \end{center}
    
    \item \texttt{flushleft} do wyrównania tekstu do lewej strony:
    \begin{flushleft}
        Ten tekst jest wyrównany do lewej.
    \end{flushleft}
    
    \item \texttt{flushright} do wyrównania tekstu do prawej strony:
    \begin{flushright}
        Ten tekst jest wyrównany do prawej.
    \end{flushright}
\end{itemize}

\subsection{Inne przydatne formaty}
\begin{itemize}
    \item \texttt{\textbackslash emph\{...\}} do podkreślenia ważnych słów, np. \emph{to jest bardzo ważne}.
    \item Pisanie kodu lub komend w trybie maszynopisu za pomocą \texttt{\textbackslash texttt\{...\}}, np. \texttt{print("Hello, world!")}.
\end{itemize}


%%%%%%%%%%%%%%%%%%%%%%%%%%%%%%%%%%%%%%%%%%%%%%%%%%%%%%%%%%%%%%%%%%%%%%%%%%%%%%%%%%%%%%%%%%%%%%%%%

\section{Listy}

W \LaTeX{} możemy tworzyć listy numerowane, nienumerowane oraz listy z niestandardowymi symbolami. Przykłady poniżej pokazują, jak wykorzystać różne typy list oraz jak je dostosować.

\subsection{Listy numerowane}
Listy numerowane tworzymy za pomocą środowiska \texttt{enumerate}. Numery elementów są generowane automatycznie.
\begin{enumerate}
    \item Element pierwszy.
    \item Element drugi.
    \item Element trzeci.
\end{enumerate}

\subsection{Listy nienumerowane}
Listy nienumerowane są tworzone za pomocą środowiska \texttt{itemize}. Domyślnie każdy element listy zaczyna się od punktu, ale można to zmienić, definiując inny symbol.
\begin{itemize}
    \item Element pierwszy.
    \item Element drugi.
\end{itemize}

Dostosowywanie symboli w listach nienumerowanych:
\begin{itemize}
    \item[$\star$] Symbol gwiazdki przy pierwszym elemencie.
    \item[$\diamond$] Symbol rombu przy drugim elemencie.
\end{itemize}

\subsection{Listy z niestandardowymi etykietami}
Możemy także dostosować listy do własnych potrzeb, definiując niestandardowe etykiety:
\begin{itemize}
    \item[(i)] Element pierwszy (etykieta rzymska).
    \item[(ii)] Element drugi.
\end{itemize}

\begin{itemize}
    \item[(kotek 1)] Element pierwszy (etykieta tekstowa).
    \item[(kotek 2)] Element drugi.
\end{itemize}

\subsection{Listy zagnieżdżone}
Listy można zagnieżdżać, tworząc listy wewnątrz list:
\begin{enumerate}
    \item Element pierwszy.
    \begin{itemize}
        \item Pod-element pierwszy.
        \item Pod-element drugi.
    \end{itemize}
    \item Element drugi.
\end{enumerate}

\subsection{Listy opisowe}
Listy opisowe (\texttt{description}) pozwalają na stosowanie niestandardowych opisów zamiast numerów czy kropek. Jest to przydatne przy tworzeniu definicji lub wyliczeń o charakterze opisowym:
\begin{description}
    \item[Pierwszy element] Opis pierwszego elementu.
    \item[Drugi element] Opis drugiego elementu.
\end{description}

Dzięki różnym opcjom tworzenia list, \LaTeX{} pozwala na dużą elastyczność w organizacji informacji.

%%%%%%%%%%%%%%%%%%%%%%%%%%%%%%%%%%%%%%%%%%%%%%%%%%%%%%%%%%%%%%%%%%%%%%%%%%%%%%%%%%%%%%%%%%%%%%%%%

\section{Wyrażenia matematyczne}

W tej sekcji zaprezentujemy różne sposoby formatowania wyrażeń matematycznych w \LaTeX{}.

\subsection{Definiowanie funkcji i odniesienia do równań}

Możemy zdefiniować funkcję $f_1: \mathbb{R} \to [-1, 1]$ w następujący sposób:
\begin{equation} \label{eq:sin_def}
    f_1(x) = \sin(\pi x)
\end{equation}

Korzystając ze wzorów \eqref{eq:sin_def} oraz \eqref{eq:tanh}, możemy obliczyć pierwszą i drugą pochodną funkcji $f_1(x)$:
\begin{align} \label{eq:derivatives}
    f_1'(x) &= \pi \cos(\pi x), \\
    f_1''(x) &= -\pi^2 \sin(\pi x).
\end{align}

\subsection{Macierze, wektory i operacje na nich}

W \LaTeX{} możemy definiować nie tylko macierze, ale również wektory oraz wykonywać operacje na wektorach i macierzach. Przykładowo, macierz $A$ może wyglądać następująco:
\begin{equation} \label{eq:matrix_A}
    A = 
    \begin{pmatrix}
        1 & 2 \\ 
        3 & 4
    \end{pmatrix}.
\end{equation}

Wektory można zapisać w formie kolumnowej lub wierszowej. Przykład wektora kolumnowego:
\begin{equation} \label{eq:vector_col}
    \mathbf{v} = 
    \begin{pmatrix}
        1 \\ 
        2 \\ 
        3 
    \end{pmatrix}.
\end{equation}

Przykład wektora wierszowego:
\begin{equation} \label{eq:vector_row}
    \mathbf{w} = 
    \begin{pmatrix}
        4 & 5 & 6 
    \end{pmatrix}.
\end{equation}

Oto kilka operacji na wektorach i macierzach, często stosowanych w matematyce i analizie danych.

\paragraph{Transpozycja macierzy i wektorów}
Transpozycję macierzy $A$ zapisujemy jako $A^T$:
\begin{equation} \label{eq:transpose}
    A^T = 
    \begin{pmatrix}
        1 & 3 \\ 
        2 & 4 
    \end{pmatrix}.
\end{equation}

Transpozycję wektora wierszowego $\mathbf{w}$ można zapisać jako wektor kolumnowy $\mathbf{w}^T$:
\begin{equation} \label{eq:vector_transpose}
    \mathbf{w}^T = 
    \begin{pmatrix}
        4 \\ 
        5 \\ 
        6 
    \end{pmatrix}.
\end{equation}

\paragraph{Iloczyn skalarny}
Iloczyn skalarny dwóch wektorów $\mathbf{v}$ i $\mathbf{w}$ jest definiowany jako:
\begin{equation} \label{eq:dot_product}
    \mathbf{v} \cdot \mathbf{w} = 1 \cdot 4 + 2 \cdot 5 + 3 \cdot 6 = 32.
\end{equation}

\paragraph{Iloczyn macierzy}
Iloczyn macierzy $A$ i $B$ (jeśli są zgodne wymiarowo) jest zapisany jako $A B$. Przykład:
\begin{equation} \label{eq:matrix_multiplication}
    A B = 
    \begin{pmatrix}
        1 & 2 \\ 
        3 & 4 
    \end{pmatrix}
    \begin{pmatrix}
        5 & 6 \\ 
        7 & 8 
    \end{pmatrix}
    = 
    \begin{pmatrix}
        19 & 22 \\ 
        43 & 50 
    \end{pmatrix}.
\end{equation}

\paragraph{Norma wektora}
Norma wektora $\mathbf{v}$, oznaczona jako $\|\mathbf{v}\|$, jest definiowana jako:
\begin{equation} \label{eq:vector_norm}
    \|\mathbf{v}\| = \sqrt{1^2 + 2^2 + 3^2} = \sqrt{14}.
\end{equation}
Norma jest często stosowana jako miara długości lub wielkości wektora.

\paragraph{Inne symbole i operacje}
Inne przydatne symbole matematyczne, stosowane w kontekście wektorów i macierzy, to:
\[
\det(A) \text{ - wyznacznik macierzy } A, \quad A^{-1} \text{ - macierz odwrotna do } A.
\]

\subsection{Pochodne i całki}

W \LaTeX{} mamy możliwość zapisu wielu operatorów różniczkowych i całkowych, takich jak pochodne, gradient, dywergencja, oraz różne rodzaje całek. Poniżej przedstawiamy przykłady.

\paragraph{Pochodna zwykła} Pochodna funkcji $f(x)$ względem $x$ jest zapisana jako:
\begin{equation}
    f'(x) = \frac{d}{dx} f(x).
\end{equation}

\paragraph{Pochodna cząstkowa} Pochodna cząstkowa funkcji $g(x, y)$ względem zmiennej $x$:
\begin{equation}
    \frac{\partial g}{\partial x}.
\end{equation}

\paragraph{Pochodna drugiego rzędu} Pochodna drugiego rzędu funkcji $f(x)$:
\begin{equation}
    f''(x) = \frac{d^2}{dx^2} f(x),
\end{equation}
a w przypadku pochodnej cząstkowej funkcji $g(x, y)$:
\begin{equation}
    \frac{\partial^2 g}{\partial x^2}.
\end{equation}

\paragraph{Pochodna mieszana} Pochodna mieszana drugiego rzędu funkcji $g(x, y)$ względem $x$ i $y$:
\begin{equation}
    \frac{\partial^2 g}{\partial x \partial y}.
\end{equation}

\paragraph{Gradient} Gradient funkcji skalarnej $f(x, y, z)$ jest zapisywany jako:
\begin{equation}
    \nabla f = \left( \frac{\partial f}{\partial x}, \frac{\partial f}{\partial y}, \frac{\partial f}{\partial z} \right).
\end{equation}

\paragraph{Dywergencja} Dywergencja pola wektorowego $\mathbf{F}(x, y, z) = (F_x, F_y, F_z)$:
\begin{equation}
    \nabla \cdot \mathbf{F} = \frac{\partial F_x}{\partial x} + \frac{\partial F_y}{\partial y} + \frac{\partial F_z}{\partial z}.
\end{equation}

\paragraph{Rotacja (wir)} Rotacja (wir) pola wektorowego $\mathbf{F}(x, y, z)$:
\begin{equation}
    \nabla \times \mathbf{F} = \left( \frac{\partial F_z}{\partial y} - \frac{\partial F_y}{\partial z}, \frac{\partial F_x}{\partial z} - \frac{\partial F_z}{\partial x}, \frac{\partial F_y}{\partial x} - \frac{\partial F_x}{\partial y} \right).
\end{equation}

\paragraph{Laplasjan} Laplasjan funkcji skalarnej $f(x, y, z)$, który jest równy dywergencji gradientu:
\begin{equation}
    \Delta f = \nabla^2 f = \frac{\partial^2 f}{\partial x^2} + \frac{\partial^2 f}{\partial y^2} + \frac{\partial^2 f}{\partial z^2}.
\end{equation}

\paragraph{Całka nieoznaczona}
Całka nieoznaczona funkcji $f(x)$ względem $x$:
\begin{equation}
    \int f(x) \, dx.
\end{equation}

\paragraph{Całka oznaczona}
Całka oznaczona funkcji $f(x)$ na przedziale $[a, b]$:
\begin{equation}
    \int_a^b f(x) \, dx.
\end{equation}

%%%%%%%%%%%%%%%%%%%%%%%%%%%%%%%%%%%%%%%%%%%%%%%%%%%%%%%%%%%%%%%%%%%%%%%%%%%%%%%%%%%%%%%%%%%%%%%%%

\subsection{Zbiory i logika}

\paragraph{Podstawowe operacje na zbiorach} W LaTeX możemy użyć różnych symboli do operacji na zbiorach, takich jak:
\begin{itemize}
    \item Zawieranie: $A \subset B$ oznacza, że zbiór $A$ jest podzbiorem zbioru $B$, natomiast $A \subseteq B$ to zawieranie z możliwością równości.
    \item Suma zbiorów: $A \cup B$ reprezentuje sumę zbiorów $A$ i $B$.
    \item Iloczyn zbiorów: $A \cap B$ oznacza część wspólną zbiorów $A$ i $B$.
    \item Różnica zbiorów: $A \setminus B$ to różnica zbiorów $A$ i $B$, czyli elementy należące do $A$ i nie należące do $B$.
\end{itemize}

\paragraph{Symbole specjalne} Do zapisu symboli zbiorów i funkcji możemy użyć różnych oznaczeń:
\begin{itemize}
    \item Zbiór liczb naturalnych: $\mathbb{N}$, całkowitych: $\mathbb{Z}$, wymiernych: $\mathbb{Q}$, rzeczywistych: $\mathbb{R}$ oraz zespolonych: $\mathbb{C}$.
    \item Przedział domknięty i otwarty: $[a, b]$ oznacza przedział domknięty, natomiast $(a, b)$ przedział otwarty.
\end{itemize}

\paragraph{Kwantyfikatory} W logice matematycznej kwantyfikatory są podstawowymi symbolami i w LaTeX zapisujemy je w następujący sposób:
\begin{itemize}
    \item Kwantyfikator ogólny (uniwersalny): $\forall x \in A$ oznacza „dla każdego $x$ należącego do $A$”.
    \item Kwantyfikator szczególny (egzystencjalny): $\exists x \in A$ oznacza „istnieje $x$ należący do $A$”.
\end{itemize}

\paragraph{Implikacja i równoważność} W LaTeX dostępne są specjalne symbole dla wyrażeń logicznych:
\begin{itemize}
    \item Implikacja: $A \implies B$ oznacza, że „jeśli $A$, to $B$”.
    \item Równoważność: $A \iff B$ oznacza, że „$A$ jest równoważne $B$”.
\end{itemize}

\paragraph{Prawa zbiorów} Możemy również zapisać różne prawa zbiorów, takie jak prawo De Morgana:
\begin{equation}
    \overline{A \cup B} = \overline{A} \cap \overline{B},
\end{equation}
oraz prawo przemienności sumy i iloczynu:
\begin{equation}
    A \cup B = B \cup A, \quad A \cap B = B \cap A.
\end{equation}

\paragraph{Zapis logiczny w matematyce} W zapisie matematycznym możemy stosować symbole, które oddają różne relacje i własności:
\begin{itemize}
    \item Równość: $=$, nierówność: $\neq$, mniejsze lub równe: $\leq$, większe lub równe: $\geq$.
    \item Podzielność: $a \mid b$ oznacza, że $a$ dzieli $b$.
\end{itemize}

%%%%%%%%%%%%%%%%%%%%%%%%%%%%%%%%%%%%%%%%%%%%%%%%%%%%%%%%%%%%%%%%%%%%%%%%%%%%%%%%%%%%%%%%%%%%%%%%%

\subsection{Ciągi, sumy i szeregi}

\paragraph{Ciągi liczbowe} W LaTeX możemy zapisywać ciągi liczbowe, wskazując ich indeksy dolne i górne:
\begin{equation}
    a_1, a_2, \ldots, a_n, \ldots
\end{equation}
Ciąg nieskończony jest zwykle reprezentowany przez wyrażenie $\{a_n\}_{n=1}^{\infty}$, co oznacza, że $n$ rośnie do nieskończoności.

\paragraph{Sumy skończone} Symbol sumy skończonej jest oznaczany jako $\sum$. Na przykład, suma pierwszych $n$ wyrazów ciągu $a_n$ może być zapisana jako:
\begin{equation}
    S_n = \sum_{k=1}^{n} a_k.
\end{equation}
W powyższym zapisie $\sum_{k=1}^{n}$ oznacza, że dodajemy elementy ciągu $a_k$ od indeksu $k=1$ do $k=n$.

\paragraph{Sumy nieskończone} Dla szeregów nieskończonych, zapisujemy sumę od $k=1$ do nieskończoności. Przykład szeregu geometrycznego wygląda następująco:
\begin{equation}
    \sum_{k=0}^{\infty} r^k = \frac{1}{1 - r} \quad \text{dla } |r| < 1.
\end{equation}
Szereg harmoniczny to kolejny przykład szeregu nieskończonego:
\begin{equation}
    \sum_{k=1}^{\infty} \frac{1}{k}.
\end{equation}

\paragraph{Iloczyny skończone i nieskończone} W LaTeX dostępny jest również symbol iloczynu $\prod$, który można stosować do wyrażania iloczynów skończonych i nieskończonych. Przykład iloczynu skończonego:
\begin{equation}
    P_n = \prod_{k=1}^{n} a_k,
\end{equation}
gdzie mnożymy wszystkie wyrazy ciągu $a_k$ od $k=1$ do $k=n$. W przypadku iloczynu nieskończonego można zapisać:
\begin{equation}
    \prod_{k=1}^{\infty} \left(1 + \frac{1}{k^2}\right).
\end{equation}

\paragraph{Szeregi potęgowe} Szeregi potęgowe są reprezentowane jako suma wyrazów zależnych od kolejnych potęg zmiennej. Na przykład, szereg potęgowy dla funkcji $f(x)$ wygląda tak:
\begin{equation}
    f(x) = \sum_{k=0}^{\infty} c_k x^k,
\end{equation}
gdzie $c_k$ są współczynnikami szeregu.

\paragraph{Szeregi Taylora i Maclaurina} Szereg Taylora dla funkcji $f(x)$ rozwiniętej wokół punktu $x=a$ jest wyrażony jako:
\begin{equation}
    f(x) = \sum_{k=0}^{\infty} \frac{f^{(k)}(a)}{k!} (x - a)^k.
\end{equation}
Jeśli $a=0$, szereg Taylora nazywamy szeregiem Maclaurina i zapisujemy go jako:
\begin{equation}
    f(x) = \sum_{k=0}^{\infty} \frac{f^{(k)}(0)}{k!} x^k.
\end{equation}

%%%%%%%%%%%%%%%%%%%%%%%%%%%%%%%%%%%%%%%%%%%%%%%%%%%%%%%%%%%%%%%%%%%%%%%%%%%%%%%%%%%%%%%%%%%%%%%%%

\section{Tabelki}

\paragraph{Tabelka bez krawędzi} Tabelka bez krawędzi jest najprostszym rodzajem tabeli, w której nie używamy linii oddzielających komórki ani wiersze. Przykład takiej tabelki:

\begin{center}
    \begin{tabular}{c c c}
         A & B & C \\
         1 & 2 & 3
    \end{tabular}
\end{center}

W tym przypadku c oznacza, że zawartość każdej kolumny jest wyśrodkowana.

\paragraph{Tabelka z krawędziami poziomymi} Możemy dodać krawędzie poziome, aby oddzielić wiersze. Krawędzie poziome dodaje się za pomocą komendy \texttt{hline}. Przykład tabelki z krawędziami poziomymi:

\begin{center}
    \begin{tabular}{c c c}
    \hline
         A & B & C \\
         1 & 2 & 3 \\
    \hline
    \end{tabular}
\end{center}

Tutaj dodaliśmy linie na górze i na dole tabeli, aby oddzielić wiersze od reszty tekstu.

\paragraph{Tabelka z krawędziami poziomymi oraz pionowymi} Tabelki mogą mieć również krawędzie pionowe, które oddzielają kolumny. Aby dodać krawędzie pionowe, należy użyć symbolu | w definicji kolumn. Przykład tabelki z krawędziami pionowymi i poziomymi:

\begin{center}
    \begin{tabular}{|c | c | c|}
    \hline
         A & B & C \\
    \hline
         1 & 2 & 3 \\
    \hline
    \end{tabular}
\end{center}

W tym przykładzie dodaliśmy pionowe krawędzie między kolumnami i poziome krawędzie na górze, pomiędzy wierszami oraz na dole tabeli.

\paragraph{Wyrównanie tekstu w komórkach} Warto również zauważyć, że w deklaracji tabeli \texttt{begin{tabular}{c c c}} używamy liter c, l, i r, które oznaczają wyrównanie tekstu w kolumnach:
- c - wyśrodkowanie,
- l - wyrównanie do lewej,
- r - wyrównanie do prawej.

Na przykład, jeśli chcemy wyśrodkować zawartość pierwszej kolumny, wyrównać do lewej drugą i do prawej trzecią, używamy:

\begin{center}
    \begin{tabular}{c l r}
        Aaaaaaaaa & Bbbbbbbb & Ccccccccc \\
        1 & 2 & 3
    \end{tabular}
\end{center}

\section{Obrazki}

Wstawianie obrazków do dokumentu w \LaTeX{} jest prostą czynnością, która pozwala na lepsze zobrazowanie omawianych treści. Aby wstawić obrazek, używamy środowiska \texttt{figure}, które pozwala również na dodanie podpisu, numeracji oraz odwołań w tekście. Poniżej znajduje się przykładowy kod:

\begin{verbatim}
\begin{figure}
    \centering
    \includegraphics[width=0.5\linewidth]{obrazki/some_stock_photo.jpg}
    \caption{Opis obrazka}
    \label{fig:enter-label}
\end{figure}
\end{verbatim}

\noindent
A oto wyjaśnienie poszczególnych elementów:

\begin{itemize}
    \item \texttt{\textbackslash begin\{figure\}} i \texttt{\textbackslash end\{figure\}} – otwierają i zamykają środowisko do wstawiania obrazka. Cały kod obrazka znajduje się między tymi dwoma komendami.
    \item \texttt{\textbackslash centering} – ustawia obrazek na środku strony (można też używać innych komend, np. \texttt{\textbackslash flushleft} dla wyrównania do lewej).
    \item \texttt{\textbackslash includegraphics[width=0.5\linewidth]\{obrazki/some\_stock\_photo.jpg\}} – wstawia obrazek o nazwie \texttt{some\_stock\_photo.jpg}, umieszczony w katalogu \texttt{obrazki/}. Opcja \texttt{width=0.5\linewidth} oznacza, że szerokość obrazka zostanie ustawiona na 50% szerokości strony. Zamiast \texttt{linewidth} można używać innych jednostek, takich jak \texttt{cm} (centymetry) lub \texttt{in} (cale).
    \item \texttt{\textbackslash caption\{Opis obrazka\}} – dodaje podpis pod obrazkiem. Jest to ważne, ponieważ pozwala czytelnikom zrozumieć, co przedstawia obrazek. Podpisy są również numerowane.
    \item \texttt{\textbackslash label\{fig:enter-label\}} – umożliwia przypisanie etykiety obrazkowi, aby później odwołać się do niego w tekście przy użyciu komendy \texttt{\textbackslash ref\{fig:enter-label\}}. Dzięki temu numeracja obrazków jest automatyczna i spójna.
\end{itemize}

\subsection*{Appendix}
\begin{itemize}
    \item Wstawianie obrazków w \LaTeX{} jest bardzo elastyczne – można zmieniać ich rozmiar, ustawiać je w różnych miejscach strony, a także dodawać dodatkowe efekty (np. obracanie, przycinanie).
    \item Można również korzystać z różnych formatów plików graficznych: \texttt{.png}, \texttt{.jpg}, \texttt{.pdf}, \texttt{.eps}. Zaleca się używanie formatu \texttt{.pdf} w przypadku obrazów wektorowych, ponieważ nie tracą one jakości przy zmianie rozmiaru.
    \item Warto pamiętać, że obrazki wstawione do dokumentu zajmują miejsce, dlatego ich wstawianie może wpłynąć na układ strony. Można stosować opcje takie jak \texttt{[h]} (tutaj), \texttt{[t]} (na górze strony) czy \texttt{[b]} (na dole strony), aby wpłynąć na rozmieszczenie obrazków.
    \item Jeśli potrzebujesz więcej kontroli nad wyglądem obrazków, np. nad ich kadrowaniem, użyj komendy \texttt{\textbackslash includegraphics[clip, trim=left bottom right top]} do przycięcia obrazu.
\end{itemize}

\begin{figure}[ht]
    \centering
    \includegraphics[width=0.5\textwidth]{obrazki/some_stock_photo.jpg}
    \caption{Caption}
    \label{fig:enter-label}
\end{figure}

\end{document}